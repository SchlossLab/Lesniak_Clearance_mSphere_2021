\documentclass[11pt,]{article}
\usepackage{lmodern}
\usepackage{amssymb,amsmath}
\usepackage{ifxetex,ifluatex}
\usepackage{fixltx2e} % provides \textsubscript
\ifnum 0\ifxetex 1\fi\ifluatex 1\fi=0 % if pdftex
  \usepackage[T1]{fontenc}
  \usepackage[utf8]{inputenc}
\else % if luatex or xelatex
  \ifxetex
    \usepackage{mathspec}
  \else
    \usepackage{fontspec}
  \fi
  \defaultfontfeatures{Ligatures=TeX,Scale=MatchLowercase}
\fi
% use upquote if available, for straight quotes in verbatim environments
\IfFileExists{upquote.sty}{\usepackage{upquote}}{}
% use microtype if available
\IfFileExists{microtype.sty}{%
\usepackage{microtype}
\UseMicrotypeSet[protrusion]{basicmath} % disable protrusion for tt fonts
}{}
\usepackage[margin=1in]{geometry}
\usepackage{hyperref}
\hypersetup{unicode=true,
            pdftitle={Clearance of Clostridioides difficile colonization is associated with antibiotic-specific bacterial changes},
            pdfborder={0 0 0},
            breaklinks=true}
\urlstyle{same}  % don't use monospace font for urls
\usepackage{graphicx,grffile}
\makeatletter
\def\maxwidth{\ifdim\Gin@nat@width>\linewidth\linewidth\else\Gin@nat@width\fi}
\def\maxheight{\ifdim\Gin@nat@height>\textheight\textheight\else\Gin@nat@height\fi}
\makeatother
% Scale images if necessary, so that they will not overflow the page
% margins by default, and it is still possible to overwrite the defaults
% using explicit options in \includegraphics[width, height, ...]{}
\setkeys{Gin}{width=\maxwidth,height=\maxheight,keepaspectratio}
\IfFileExists{parskip.sty}{%
\usepackage{parskip}
}{% else
\setlength{\parindent}{0pt}
\setlength{\parskip}{6pt plus 2pt minus 1pt}
}
\setlength{\emergencystretch}{3em}  % prevent overfull lines
\providecommand{\tightlist}{%
  \setlength{\itemsep}{0pt}\setlength{\parskip}{0pt}}
\setcounter{secnumdepth}{0}
% Redefines (sub)paragraphs to behave more like sections
\ifx\paragraph\undefined\else
\let\oldparagraph\paragraph
\renewcommand{\paragraph}[1]{\oldparagraph{#1}\mbox{}}
\fi
\ifx\subparagraph\undefined\else
\let\oldsubparagraph\subparagraph
\renewcommand{\subparagraph}[1]{\oldsubparagraph{#1}\mbox{}}
\fi

%%% Use protect on footnotes to avoid problems with footnotes in titles
\let\rmarkdownfootnote\footnote%
\def\footnote{\protect\rmarkdownfootnote}

%%% Change title format to be more compact
\usepackage{titling}

% Create subtitle command for use in maketitle
\providecommand{\subtitle}[1]{
  \posttitle{
    \begin{center}\large#1\end{center}
    }
}

\setlength{\droptitle}{-2em}

  \title{\textbf{Clearance of \emph{Clostridioides difficile}
colonization is associated with antibiotic-specific bacterial changes}}
    \pretitle{\vspace{\droptitle}\centering\huge}
  \posttitle{\par}
    \author{}
    \preauthor{}\postauthor{}
    \date{}
    \predate{}\postdate{}
  
\usepackage{booktabs}
\usepackage{longtable}
\usepackage{array}
\usepackage{multirow}
\usepackage{wrapfig}
\usepackage{float}
\usepackage{colortbl}
\usepackage{pdflscape}
\usepackage{tabu}
\usepackage{threeparttable}
\usepackage{threeparttablex}
\usepackage[normalem]{ulem}
\usepackage{makecell}
\usepackage{caption}
\usepackage{hyperref}
\usepackage{helvet} % Helvetica font
\renewcommand*\familydefault{\sfdefault} % Use the sans serif version of the font
\usepackage[T1]{fontenc}
\usepackage[labelfont=bf]{caption}

\usepackage[none]{hyphenat}

\usepackage{setspace}
\doublespacing
\setlength{\parskip}{1em}

\usepackage{lineno}

\usepackage{pdfpages}
\floatplacement{figure}{H} % Keep the figure up top of the page

\newlength{\cslhangindent}
\setlength{\cslhangindent}{1.5em}
\newenvironment{cslreferences}%
  {\setlength{\parindent}{0pt}%
  \everypar{\setlength{\hangindent}{\cslhangindent}}\ignorespaces}%
  {\par}

\begin{document}
\maketitle

\vspace{30mm}

Running title: Clearance of \emph{Clostridioides difficile} colonization

\vspace{20mm}

Nicholas A. Lesniak\(^1\), Alyxandria M. Schubert\(^1\), Hamide
Sinani\(^1\), Patrick D. Schloss\(^{1,\dagger}\)

\vspace{30mm}

\(\dagger\) To whom correspondence should be addressed:
\href{mailto:pschloss@umich.edu}{\nolinkurl{pschloss@umich.edu}}

1. Department of Microbiology and Immunology, University of Michigan,
Ann Arbor, MI

\newpage
\linenumbers

\hypertarget{abstract}{%
\subsection{Abstract}\label{abstract}}

The gut bacterial community prevents many pathogens from colonizing the
intestine. Previous studies have associated specific bacteria with
clearing \emph{Clostridioides difficile} colonization across different
community perturbations. However, those bacteria alone have been unable
to clear \emph{C. difficile} colonization. To elucidate the changes
necessary to clear colonization, we compared differences in bacterial
abundance between communities able and unable to clear \emph{C.
difficile} colonization. We treated mice with titrated doses of
antibiotics prior to \emph{C. difficile} challenge which resulted in no
colonization, colonization and clearance, or persistent colonization.
Previously, we observed that clindamycin-treated mice were susceptible
to colonization but spontaneously cleared \emph{C. difficile}.
Therefore, we investigated whether other antibiotics would show the same
result. We found reduced doses of cefoperazone and streptomycin
permitted colonization and clearance of \emph{C. difficile}. Mice that
cleared colonization had antibiotic-specific community changes and
predicted interactions with \emph{C. difficile}. Clindamycin treatment
led to a bloom in populations related to \emph{Enterobacteriaceae}.
Clearance of \emph{C. difficile} was concurrent with the reduction of
those blooming populations and the restoration of community members
related to the \emph{Porphyromonadaceae} and \emph{Bacteroides}.
Cefoperazone created a susceptible community characterized by a drastic
reduction in the community diversity, interactions, and a sustained
increase in abundance of many facultative anaerobes. Lastly, clearance
in streptomycin-treated mice was associated with the recovery of
multiple members of the \emph{Porphyromonadaceae}, with little overlap
in the specific \emph{Porphyromonadaceae} observed in the clindamycin
treatment. Further elucidation of how \emph{C. difficile} colonization
is cleared from different gut bacterial communities will improve
\emph{C. difficile} infection treatments.

\newpage

\hypertarget{importance}{%
\subsection{Importance}\label{importance}}

The community of microorganisms, known as the microbiota, in our
intestines prevents pathogens, such as \emph{C. difficile}, from
establishing themselves and causing infection. This is known as
colonization resistance. However, when a person takes antibiotics, their
gut microbiota is disturbed. This disruption allows \emph{C. difficile}
to colonize. \emph{C. difficile} infections (CDI) are primarily treated
with antibiotics, which frequently leads to recurrent infections because
the microbiota have not yet returned to a resistant state. The infection
cycle often ends when the fecal microbiota from a presumed resistant
person are transplanted into the susceptible person. Although this
treatment is highly effective, we do not understand the mechanism of
resistance. We hope to improve the treatment of CDI through elucidating
how the bacterial community eliminates \emph{C. difficile} colonization.
We found \emph{C. difficile} was able to colonize susceptible mice but
was spontaneously eliminated in an antibiotic-treatment specific manner.
These data indicate each community had different requirements for
clearing colonization. Understanding how different communities clear
colonization will reveal targets to improve CDI treatments.

\newpage

\hypertarget{introduction}{%
\subsection{Introduction}\label{introduction}}

A complex consortium of bacteria and microbes that inhabits our gut,
known as the microbiota, prevent pathogens from colonizing and causing
disease. This protection, known as colonization resistance, is mediated
through many mechanisms such as activating host immune responses,
competing for nutrients, producing antimicrobials, and contributing to
the maintenance of the mucosal barrier (1). However, perturbations to
the intestinal community or these functions opens the possibility that a
pathogen can colonize (2). For example, the use of antibiotics perturb
the gut microbiota and can lead to \emph{Clostridioides difficile}
infection (CDI).

CDI is especially problematic due to its burden on the healthcare system
(3, 4). \emph{C. difficile} can cause severe disease, such as toxic
megacolon, diarrhea, and death (5). CDI is primarily treated with
antibiotics (6). CDIs recalcitrant to antibiotics are eliminated by
restoring the community with a fecal microbiota transplant (FMT),
returning the perturbed community to a healthier protective state (7,
8). However, FMTs are not always effective against CDI and have the risk
of transferring a secondary infection (9, 10). Therefore, we need to
better understand how the microbiota clears the infection to develop
more effective treatments.

Previous research has shown that the microbiota affects \emph{C.
difficile} colonization. Mouse models have identified potential
mechanisms of colonization resistance such as bile salt metabolism and
nutrient competition (11--14). However, studies that have restored those
functions were unable to restore complete resistance (15, 16). This
could be attributed to the complexity of the community and the
mechanisms of colonization resistance (17, 18). We previously showed
that when \emph{C. difficile} colonizes different antibiotic-treated
murine communities it modifies its metabolism to fit each specific
environment (14, 19, 20). Therefore, we have investigated the bacterial
community dynamics concurrent with \emph{C. difficile} elimination
across uniquely perturbed communities.

Jenior et al.~(20) observed that clindamycin-treated mice spontaneously
cleared \emph{C. difficile} colonization whereas mice treated with
cefoperazone and streptomycin did not. Here, we continued to explore the
different effects these three antibiotics have on \emph{C. difficile}
colonization. The purpose of this study was to elucidate the gut
bacterial community changes concurrent with elimination of \emph{C.
difficile} colonization. We hypothesized that each colonized community
has perturbation-specific susceptibilities and requires specific changes
to clear the pathogen. To induce a less severe perturbation, we reduced
the doses of cefoperazone and streptomycin. This resulted in communities
that were initially colonized to a high level (\textgreater{}\(10^{6}\)
CFU/g feces) and then spontaneously cleared \emph{C. difficile}. We
found each antibiotic resulted in unique changes in the microbiota that
were associated with the persistence or clearance of \emph{C.
difficile}. These data further support the hypothesis that \emph{C.
difficile} can exploit numerous niches in perturbed communities.

\hypertarget{results}{%
\subsection{Results}\label{results}}

\textbf{Reduced doses of cefoperazone and streptomycin allowed
communities to spontaneously clear \emph{C. difficile} colonization.} To
understand the dynamics of colonization and clearance of \emph{C.
difficile}, we first identified conditions which would allow
colonization and clearance. Beginning with clindamycin, mice were
treated with an intraperitoneal injection of clindamycin (10 mg/kg) one
day prior to challenge with \emph{C. difficile}. All mice (N = 11) were
colonized to a high level (median CFU = \(3.07 \times 10^7\)) the next
day and cleared the colonization within 10 days; 6 mice cleared \emph{C.
difficile} within 6 days (Figure 1A). Previous \emph{C. difficile}
infection models using cefoperazone and streptomycin have not
demonstrated clearance. So we next explored whether cefoperazone and
streptomycin could permit colonization and subsequent clearance with
lower doses. We began with replicating the previously established
\emph{C. difficile} infection models using these antibiotics (20). We
treated mice with cefoperazone or streptomycin in their drinking water
for 5 days (0.5 mg/mL and 5 mg/mL, respectively) and then challenged
them with \emph{C. difficile}. For both antibiotics, \emph{C. difficile}
colonization was maintained for the duration of the experiment as
previously demonstrated (Figure 1B-C) (20). Then we repeated the
\emph{C. difficile} challenge with reduced doses of the antibiotics
(cefoperazone - 0.3 and 0.1 mg/mL; streptomycin - 0.5 and 0.1 mg/mL).
For both antibiotic treatments, the lowest dose resulted in either no
colonization (N = 8) or a transient, low level colonization (N = 8,
median length = 1 day, median CFU/g = \(2.8 \times 10^3\)) (Figure
1B-C). The intermediate dose of both antibiotics resulted in a high
level colonization (median CFU/g = \(3.5 \times 10^6\)) and half (N = 8
of 16) of the mice clearing the colonization within 10 days. Based on
our previous research, which showed each of these antibiotics uniquely
changed the microbiota, we hypothesized that the microbiota varied
across these antibiotic treatments that resulted in colonization
clearance.

\textbf{Clearance of \emph{C. difficile} was associated with
antibiotic-specific changes to the microbiota.} Beginning with the
clindamycin-treated mice, we analyzed their fecal 16S rRNA gene
sequences to identify the community features related to \emph{C.
difficile} colonization and clearance. First, we compared the most
abundant bacterial genera of the communities at the time of \emph{C.
difficile} challenge. The clindamycin-treated mice became dominated by
relatives of \emph{Enterobacteriaceae} with a concurrent reduction in
the other abundant genera, except for populations of
\emph{Lactobacillus} (Figure 1D, S1). These community changes permitted
\emph{C. difficile} to colonize all of these mice, but all of the mice
were are also able to clear the colonization. We next investigated how
the microbiota diversity related to \emph{C. difficile} clearance.
Clindamycin treatment decreased the alpha diversity (\emph{P}
\textless{} 0.05) and similarity to the pre-clindamycin community at the
time of \emph{C. difficile} challenge (\emph{P} \textless{} 0.05)
(Figure 2A). But it was not necessary to restore the community
similarity to its initial state to clear \emph{C. difficile}. Therefore
we investigated the temporal differences in the abundance of the
operational taxonomic units (OTUs) between the initial untreated
community and post-clindamycin treatment at the time of challenge and
between the time of challenge and the end of the experiment. Clindamycin
treatment resulted in large decreases in 21 OTUs and a bloom of
relatives of \emph{Enterobacteriaceae} (Figure 4A). With the elimination
of \emph{C. difficile}, we observed a drastic reduction of the relatives
of \emph{Enterobacteriaceae} and recovery of 10 populations related to
\emph{Porphyromonadaceae}, \emph{Bacteroides}, \emph{Akkermansia},
\emph{Lactobacillus}, \emph{Bifidobacterium}, \emph{Lachnospiraceae},
and \emph{Clostridiales} (Figure 4A). Thus, clindamycin reduced most of
the natural community allowing \emph{C. difficile} to colonize. The
recovery of only a small portion of the community was associated with
eliminating the \emph{C.difficile} population.

We applied the same analysis to the cefoperazone-treated mice to
understand what community features were relevant to clearing \emph{C.
difficile}. Increasing the dose of cefoperazone shifted the dominant
community members from relatives of the \emph{Porphyromonadaceae},
\emph{Bacteroides} and \emph{Akkermansia} to relatives of the
\emph{Lactobacillus} and \emph{Enterobacteriaceae} at the time of
challenge (Figure 1E, S1). We saw a similar increase in relatives of
\emph{Enterobacteriaceae} with clindamycin. However, the
cefoperazone-treated mice that had larger increases in
\emph{Enterobacteriaceae} were unable to clear \emph{C. difficile}. We
next investigated the differences between the cefoperazone-treated mice
that cleared \emph{C. difficile} to those that did not. For the
communities that cleared \emph{C.difficile}, diversity was maintained
throughout the experiment (Figure 2B). The mice treated with
cefoperazone that remained colonized experienced an increase in alpha
diversity, likely driven by the decrease in highly abundant populations
and increase in low abundant populations (Figure 1E). These persistently
colonized communities also had a large shift away from the initial
community structure caused by the antibiotic treatment (\emph{P}
\textless{} 0.05), which remained through the end of the experiment
(\emph{P} \textless{} 0.05) (Figure 2B). These data suggested that it
was necessary for cefoperazone-treated mice to become more similar to
the initial pre-antibiotic community structure to clear \emph{C.
difficile}. We next investigated the changes in OTU abundances between
the communities that cleared \emph{C. difficile} and those that did not
to elucidate the community members involved in clearance. Communities
that remained colonized were significantly enriched in facultative
anaerobic populations including \emph{Enterococcus}, \emph{Pseudomonas},
\emph{Staphylococcus}, and \emph{Enterobacteriaceae} at the time of
challenge. Communities that cleared \emph{C. difficile} had significant
enrichment in 10 different OTUs related to the \emph{Porphyromonadaceae}
at the end of the experiment (Figure 3A). We were also interested in the
temporal changes within each community so we investigated which OTUs
changed due to antibiotic treatment or during the \emph{C. difficile}
colonization. The majority of significant temporal differences in OTUs
for cefoperazone-treated mice occurred in persistently colonized
communities. Persistently colonized communities had a persistent loss of
numerous relatives of the \emph{Porphyromonadaceae} and increases in the
relative abundance of facultative anaerobes (Figure 4C, S2). Overall,
persistent \emph{C. difficile} colonization in cefoperazone-treated mice
was associated with a shift in the microbiota to a new community
structure which seemed unable to recover from the antibiotic
perturbation, whereas clearance occurred when the community was capable
of returning to its original structure.

Finally, we identified the differences in \emph{C. difficile}
colonization for streptomycin-treated mice. Increasing the dose of
streptomycin maintained the abundance of relatives of the
\emph{Porphyromonadaceae} and \emph{Bacteroides}, but reduced most of
the other genera including populations of the \emph{Lactobacillus},
\emph{Lachnospiraceae}, \emph{Ruminococcaceae}, \emph{Alistipes}, and
\emph{Clostridiales} (Figure 1F). Both communities that cleared and
those that remained colonized had similar changes in diversity.
Streptomycin-treated mice became mildly dissimilar (\emph{P} \textless{}
0.05) and less diverse (\emph{P} \textless{} 0.05) with streptomycin
treatment but by the end of the experiment returned to resemble the
pre-antibiotic community (\emph{P} \textless{} 0.05) (Figure 2C). Those
communities that remained colonized had slightly lower alpha-diversity
than those that cleared \emph{C. difficile}. (\emph{P} \textless{}
0.05). Persistently colonized mice had reduced relative abundance of
relatives of \emph{Alistipes}, \emph{Anaeroplasma}, and
\emph{Porphyromonadaceae} at time of challenge compared to the mice that
cleared \emph{C. difficle} (Figure 3B). At the end of the experiment the
mice that were still colonized had lower abundances of
\emph{Turicibacter}, \emph{Alistipes}, and \emph{Lactobacillus}. Since
most of the differences were reduced relative abundances in the
colonized mice, we were interested to explore what temporal changes
occurred between pre-antibiotic treatment, the time of challenge, and
the end of the experiment for the communities that cleared \emph{C.
difficile}. The temporal changes in streptomycin-treated mice were more
subtle than those observed with the other antibiotic treatments. At the
time of challenge, the communities that remained colonization had
reductions in 4 OTUs related to the \emph{Porphyromonadaceae}. Those
that cleared \emph{C. difficile} also had changes in OTUs related to the
\emph{Porphyromonadaceae}, however, 2 populations decreased and 2
increased in abundance (Figure 4B, D). At the end of the experiment, all
communities experienced recovery of the abundance of many of the
populations changed by the streptomycin treatment, but the communities
that remained colonized did not recover 5 of the OTUs of
\emph{Alistipes}, \emph{Lactobacillus}, and \emph{Porphyromonadaceae}
that were reduced by streptomycin. The differences between the
streptomycin-treated mice that remained colonized and those had been
cleared of \emph{C. difficile} were not as distinct as those observed
with the cefoperazone treatment. The differences between colonized and
cleared streptomycin-treated mice were minimal, which suggested the few
differences may be responsible for the clearance. Overall, these data
revealed that while there were commonly affected families across the
antibiotic treatments, such as the \emph{Porphyromonadaceae}, \emph{C.
difficile} clearance was associated with community and OTU differences
specific to each antibiotic.

\textbf{Distinct features of the bacterial community at the time of
infection predicted end point colonization.} To determine whether the
community composition at the time of \emph{C. difficile} challenge could
predict \emph{C. difficile} clearance, we built a machine learning model
using L2 logistic regression. We evaluated the predictive performance of
the model using the area under the receiver operating characteristic
curve (AUROC), where a value of 0.5 indicated the model is random and
1.0 indicated the model always correctly predicts the outcome. Our model
resulted in a AUROC of 0.986 {[}IQR 0.970-1.000{]}, which suggested that
the model was able to use the relative abundance of OTUs at the time of
challenge to accurately predict colonization clearance (Figure S3). To
assess the important features, we randomly permuted each OTU feature by
removing it from the training set to determine its effect on the
prediction (Figure 5A). The most important feature was an OTU related to
the \emph{Enterobacteriaceae}, whose abundance predicted clearance. This
result appears to have been strongly driven by the clindamycin data
(Figure 5B, C). The remaining OTU features did not have a large effect
on the model performance, which suggested that the model decision was
spread across many features. These results revealed the model used the
relative abundance data of the community members and the relationship
between those abundances to correctly classify clearance. There were
many OTUs with treatment and outcome specific abundance patterns that
did not agree with the odds ratio of the OTU used by the model. For
example, \emph{Enterobacteriaceae} abundance influenced the model to
predict clearance (Figure 5B), however in experiments that used
cefoperazone, the communities that remained colonized had higher
abundances of \emph{Enterobacteriaceae} than the communities that
cleared colonization (Figure 5C). The model arrived at the correct
prediction through the influence of other OTUs. Therefore, the model
used different combinations of multiple OTUs and their relative
abundances across treatments to predict \emph{C. difficile} clearance.
These data can offer a basis for hypotheses regarding the distinct
combinations of bacteria that promote \emph{C. difficile} clearance.

\textbf{Conditional independence networks revealed treatment-specific
relationships between the community members and \emph{C. difficile}
during colonization clearance.} We next investigated the relationship
between temporal changes in the community and \emph{C. difficile} by
building a conditional independence network for each treatment using
SPIEC-EASI (sparse inverse covariance estimation for ecological
association inference) (21). First, we focused on the first-order
associations of \emph{C. difficile} (Figure 6A). In clindamycin-treated
mice, \emph{C. difficile} had positive associations with relatives of
\emph{Enterobacteriaceae}, \emph{Pseudomonas}, and \emph{Olsenella} and
negative associations with relatives of the \emph{Lachnospiraceae} and
\emph{Clostridium} XIVa. \emph{C. difficile} had limited associations in
cefoperazone-treated mice; the primary association was positive with
relatives of \emph{Enterobacteriaceae}. In streptomycin-treated mice,
\emph{C. difficile} had negative associations with relatives of the
\emph{Porphyromonadaceae} and positive associations with populations of
the \emph{Ruminococcaceae}, \emph{Bacteroidetes}, \emph{Clostridium} IV
and \emph{Olsenella}. Next, we quantified the degree centrality, the
number of associations between each OTU for the whole network of each
antibiotic and outcome, and betweenness centrality, the number of
associations connecting two OTUs that pass through an OTU (Figure 6B).
This analysis revealed cefoperazone treatment resulted in networks
primarily composed of singular associations with much lower degree
centrality (\emph{P} \textless{} 0.05) and betweenness centrality
(\emph{P} \textless{} 0.05) than the other antibiotic treatments.
Communities that were treated with cefoperazone that resulted in cleared
or persistent colonization had 10 to 100-fold lower betweenness
centrality values than communities treated with clindamycin or
streptomycin. Collectively, these networks suggest \emph{C. difficile}
colonization was affected by unique sets of OTUs in mice treated with
clindamycin and streptomycin, but cefoperazone treatment eliminated
bacteria critical to maintaining community interactions and had few
populations that associated with \emph{C. difficile}.

\hypertarget{discussion}{%
\subsection{Discussion}\label{discussion}}

We have shown that different antibiotic treatments resulted in specific
changes to the microbiota that were associated with \emph{C. difficile}
clearance. Clindamycin-treated mice became susceptible with a dominant
bloom in populations related to \emph{Enterobacteriaceae}. Clearance was
associated with the resolution of the bloom and recovery of bacteria
that were reduced by the antibiotic treatment. Cefoperazone-treated mice
became susceptible with the expansion of numerous facultative anaerobes.
Communities with a sustained presence of these facultative anaerobes
were unable to recover from the initial antibiotic perturbation or clear
the colonization, whereas the communities that returned to their initial
community were able to clear \emph{C. difficile} colonization.
Streptomycin-treated mice became susceptible with fewer and smaller
changes than the other treatments. The communities that cleared
colonization had slightly higher \(\alpha\)-diversity than those that
remained colonized. Additionally, all communities in mice treated with
streptomycin had similar numbers of OTUs changing through the experiment
but the specific OTUs were different for each outcome. These
observations support our hypothesis that each colonized community has
antibiotic-specific changes that create unique conditions for \emph{C.
difficile} colonization and requires specific changes within each
community to clear \emph{C. difficile}.

Previous studies have identified microbiota associated with \emph{C.
difficile} colonization resistance in either a set of closely related
murine communities or collectively across many different susceptible
communities (11, 15, 22). These bacteria were then tested in \emph{C.
difficile} infection models. These experiments were able to show
decreased colonization but were unable to fully clear \emph{C.
difficile} (11, 23). Rather than looking for similarities across all
susceptible communities, we explored the changes that were associated
with \emph{C. difficile} clearance for each antibiotic. Even though
these mice all came from the same breeding colony with similar initial
microbiomes, \emph{C. difficile} clearance was associated with
antibiotic-specific changes in community diversity, OTU abundances, and
associations between OTUs. Our data suggest that the set of bacteria
necessary to restore colonization resistance following one antibiotic
perturbation may not be effective for all antibiotic perturbations. We
have developed this modeling framework starting from a single mouse
community. It should also be relevant when considering interpersonal
variation among humans (24).

Recent studies have begun to uncover how communities affect \emph{C.
difficile} colonization (17--20, 24). We attempted to understand the
general trends in each antibiotic treatment that lead to clearance of
\emph{C. difficile}. We categorized the general changes and microbial
relationships of these experiments into three models. First, a model of
temporary opportunity characterized by the transient dominance of a
facultative anaerobe which permits \emph{C. difficile} colonization but
\emph{C. difficile} is not able to persist, as with clindamycin
treatment. We hypothesize this susceptibility is due to a transient
repression of community members and interventions which further perturb
the community may worsen the infection. Time alone may be sufficient for
the community to clear colonization (15, 22, 25) but treating the
community with an antibiotic or the bowel preparation for an FMT (26,
27) may prolong susceptibility by eliminating protective functions or
opening new niches. Second, a model of an extensive opportunity
characterized by a significant perturbation leading to a persistent
increase in facultative anaerobes and exposing multiple niches, as with
cefoperazone treatment. These communities appear to have been severely
depleted of multiple critical community members and are likely lacking
numerous protective functions (20). We hypothesize multiple niches are
available for \emph{C. difficile} to colonize. In this scenario, a full
FMT may be insufficient to provide adequate diversity and abundance to
outcompete and occupy all the exposed niches. Multiple FMTs (28, 29) or
transplant of an enriched fecal community (30) may be necessary to
recover the microbiota enough to outcompete \emph{C. difficile} for the
nutrient niches and replace the missing protective functions. Third, a
model of a specific opportunity characterized by a perturbation that
only affects a select portion of the microbiota, leading to small
changes in relative abundance and a slight decrease in diversity,
opening a limited niche for \emph{C. difficile} to colonize, as with
streptomycin treatment. We hypothesize that a few specific bacteria
would be necessary to recolonize the exposed niche space and eliminate
\emph{C. difficile} colonization (13, 17). A fecal microbiota transplant
may contain the bacterial diversity needed to fill the open niche space
and help supplant \emph{C. difficile} from the exposed niche of the
colonized community. Analyzing each of these colonization models
individually allowed us to understand how each may clear \emph{C.
difficile} colonization.

Future investigations can further identify the exposed niches of
susceptible communities and the requirements to clear \emph{C.
difficile} colonization. One common theme for susceptibility across
treatments was the increased abundance of facultative anaerobes. These
blooms of facultative anaerobes could be attributed to the loss of the
indigenous obligate anaerobes with antibiotic treatment (31, 32).
However, it is unclear what prevents the succession from the facultative
anaerobes back to the obligate anaerobes in cefoperazone-treated mice.
Future studies should investigate the relationship between facultative
anaerobe blooms and susceptibility to colonization as well as
interventions to recover the obligate anaerobes. Another aspect to
consider in future experiments is \emph{C. difficile} strain
specificity. Other strains may fill different niche space and fill other
community interactions (33--35). For example, more virulent strains,
like \emph{C. difficile} VPI 10463, may have a greater effect on the gut
environment since it produces more toxin (15, 36). Those differences
could have different impacts on the susceptible community and change the
requirements to clear \emph{C. difficile}. Finally, we have shown that
the functions found in communities at peak colonization are
antibiotic-specific (20). Here, we have shown the community changes
associated with \emph{C. difficile} clearance are antibiotic-specific.
It is unknown how the community functions contributing to \emph{C.
difficile} clearance compare across antibiotics. Examining the changes
in transcription and metabolites during clearance will help define the
activities necessary to clear \emph{C. difficile} and if they are
specific to the perturbation. This information will build upon the
community differences presented in this study and move us closer to
elucidating how the microbiota clears \emph{C. difficile} colonization
and developing targeted therapeutics.

We have shown that mice became susceptible to \emph{C. difficile}
colonization after three different antibiotic treatments and then
differed in their ability to clear the colonization. These experiments
have shown that each antibiotic treatment resulted in different
community changes leading to \emph{C. difficile} clearance. These
differences suggest that a single mechanism of infection and one
treatment for all \emph{C. difficile} infections may not be appropriate.
While our current use of FMT to eliminate CDI is highly effective, it
does not work in all patients and has even resulted in adverse
consequences (7--10). The findings in this study may help explain why
FMTs may be ineffective. Although an FMT transplants a whole community,
it may not be sufficient to replace the missing community members or
functions to clear \emph{C. difficile}. Alternatively, the FMT procedure
itself may disrupt the natural recovery of the community. The knowledge
of how a community clears \emph{C. difficile} colonization will advance
our ability to develop targeted therapies to manage CDI.

\hypertarget{materials-and-methods}{%
\subsection{Materials and Methods}\label{materials-and-methods}}

\textbf{Animal care.} All mice were obtained from a single breeding
colony and maintained in specific-pathogen-free (SPF) conditions at the
University of Michigan animal facility. All mouse protocols and
experiments were approved by the University Committee on Use and Care of
Animals at the University of Michigan and completed in agreement with
approved guidelines.

\textbf{Antibiotic administration.} Mice were given one of three
antibiotics, cefoperazone, clindamycin, or streptomycin. Cefoperazone
(0.5, 0.3, or 0.1 mg/ml) and streptomycin (5, 0.5, or 0.1 mg/ml) were
delivered via drinking water for 5 days. Clindamycin (10 mg/kg) was
administered through intraperitoneal injection.

\textbf{\emph{C. difficile} challenge.} Mice were returned to untreated
drinking water for 24 hours before challenging with \emph{C. difficile}
strain 630\(\Delta\)erm spores. \emph{C. difficile} spores were
aliquoted from a single spore stock stored at 4\(^\circ\)C. Spore
concentration was determined one week prior to the day of challenge
(37). \(10^{3}\) \emph{C. difficile} spores were orally gavaged into
each mouse. Once the gavages were completed, the remaining spore
solution was serially diluted and plated to confirm the spore
concentration that was delivered.

\textbf{Sample collection.} Fecal samples were collected on the day
antibiotic treatment was started, on the day of \emph{C. difficile}
challenge and the following 10 days. For the day of challenge and
beyond, a fecal sample was also collected and weighed. Under anaerobic
conditions a fecal sample was serially diluted in anaerobic
phosphate-buffered saline and plated on TCCFA plates. After 24 hours of
anaerobic incubation at 37\(^\circ\)C, the number of colony forming
units (CFU) were determined (38).

\textbf{DNA sequencing.} Total bacterial DNA was extracted from each
fecal sample using MOBIO PowerSoil-htp 96-well soil DNA isolation kit.
We created amplicons of the 16S rRNA gene V4 region and sequenced them
using an Illumina MiSeq as described previously (39).

\textbf{Sequence curation.} Sequences were processed using
mothur(v.1.43.0) as previously described (39). Briefly, we used a 3\%
dissimilarity cutoff to group sequences into operational taxonomic units
(OTUs). We used a naive Bayesian classifier with the Ribosomal Database
Project training set (version 16) to assign taxonomic classifications to
each OTU (41). With the fecal samples, we also sequenced a mock
community with a known community composition and their true 16s rRNA
gene sequences. We processed this mock community along with our samples
to determine our sequence curation resulted in an error rate of 0.019\%.

\textbf{Statistical analysis and modeling.} Diversity comparisons were
calculated in mothur. To compare \(\alpha\)-diversity metrics, we
calculated the number of OTUs (S\textsubscript{obs}) and the Inverse
Simpson diversity index. To compare across communities, we calculated
dissimilarity matrices based on metric of Yue and Clayton (42). All
calculations were made by rarifying samples to 1,200 sequences per
sample to limit biases due to uneven sampling. OTUs were subsampled to
1,200 counts per sample and remaining statistical analysis and data
visualization was performed in R (v3.5.1) with the tidyverse package
(v1.3.0). Significance of pairwise comparisons of \(\alpha\)-diversity
(S\textsubscript{obs} and Inverse Simpson), \(\beta\)-diversity
(\(\theta\)\textsubscript{YC}), OTU abundance, and network centrality
(betweenness and degree) were calculated by pairwise Wilcoxon rank sum
test and then \emph{P} values were corrected for multiple comparisons
with a Benjamini and Hochberg adjustment for a type I error rate of 0.05
(43). Logistic regression models were constructed with OTUs from all day
0 samples using half of the samples to train and the other half to test
the model. The model was developed from the caret R package (v6.0-85)
and previously developed machine learning pipeline (44). For each
antibiotic treatment, conditional independence networks were calculated
from the day 1 through 10 samples of all mice initially colonized using
SPIEC-EASI (sparse inverse covariance estimation for ecological
association inference) methods from the SpiecEasi R package after
optimizing lambda to 0.001 with a network stability between 0.045 and
0.05 (v1.0.7) (21). Network centrality measures degree and betweenness
were calculated on whole networks using functions from the igraph R
package (v1.2.4.1).

\textbf{Code availability.} Scripts necessary to reproduce our analysis
and this paper are available in an online repository
(\url{https://github.com/SchlossLab/Lesniak_Clearance_XXXX_2020}).

\textbf{Sequence data accession number.} All 16S rRNA gene sequence data
and associated metadata are available through the Sequence Read Archive
via accession PRJNA674858.

\hypertarget{acknowledgements}{%
\subsection{Acknowledgements}\label{acknowledgements}}

Thank you to Begüm Topçuoglu and Sarah Tomkovich for critical discussion
in the development and execution of this project. This work was
supported by several grants from the National Institutes for Health
R01GM099514, U19AI090871, U01AI12455, and P30DK034933. Additionally, NAL
was supported by the Molecular Mechanisms of Microbial Pathogenesis
training grant (NIH T32 AI007528). The funding agencies had no role in
study design, data collection and analysis, decision to publish, or
preparation of the manuscript.

\newpage

\hypertarget{references}{%
\subsection{References}\label{references}}

\hypertarget{refs}{}
\begin{cslreferences}
\leavevmode\hypertarget{ref-ducarmon2019}{}%
1. \textbf{Ducarmon QR}, \textbf{Zwittink RD}, \textbf{Hornung BVH},
\textbf{Schaik W van}, \textbf{Young VB}, \textbf{Kuijper EJ}. 2019. Gut
microbiota and colonization resistance against bacterial enteric
infection. Microbiology and Molecular Biology Reviews \textbf{83}.
doi:\href{https://doi.org/10.1128/mmbr.00007-19}{10.1128/mmbr.00007-19}.

\leavevmode\hypertarget{ref-britton2012}{}%
2. \textbf{Britton RA}, \textbf{Young VB}. 2012. Interaction between the
intestinal microbiota and host in clostridium difficile colonization
resistance. Trends in Microbiology \textbf{20}:313--319.
doi:\href{https://doi.org/10.1016/j.tim.2012.04.001}{10.1016/j.tim.2012.04.001}.

\leavevmode\hypertarget{ref-lessa2015}{}%
3. \textbf{Lessa FC}, \textbf{Mu Y}, \textbf{Bamberg WM},
\textbf{Beldavs ZG}, \textbf{Dumyati GK}, \textbf{Dunn JR},
\textbf{Farley MM}, \textbf{Holzbauer SM}, \textbf{Meek JI},
\textbf{Phipps EC}, \textbf{Wilson LE}, \textbf{Winston LG},
\textbf{Cohen JA}, \textbf{Limbago BM}, \textbf{Fridkin SK},
\textbf{Gerding DN}, \textbf{McDonald LC}. 2015. Burden of Clostridium
difficile Infection in the united states. New England Journal of
Medicine \textbf{372}:825--834.
doi:\href{https://doi.org/10.1056/nejmoa1408913}{10.1056/nejmoa1408913}.

\leavevmode\hypertarget{ref-zimlichman2013}{}%
4. \textbf{Zimlichman E}, \textbf{Henderson D}, \textbf{Tamir O},
\textbf{Franz C}, \textbf{Song P}, \textbf{Yamin CK}, \textbf{Keohane
C}, \textbf{Denham CR}, \textbf{Bates DW}. 2013. Health careAssociated
infections. JAMA Internal Medicine \textbf{173}:2039.
doi:\href{https://doi.org/10.1001/jamainternmed.2013.9763}{10.1001/jamainternmed.2013.9763}.

\leavevmode\hypertarget{ref-spigaglia2016}{}%
5. \textbf{Spigaglia P}, \textbf{Barbanti F}, \textbf{Morandi M},
\textbf{Moro ML}, \textbf{Mastrantonio P}. 2016. Diagnostic testing for
clostridium difficile in italian microbiological laboratories. Anaerobe
\textbf{37}:29--33.
doi:\href{https://doi.org/10.1016/j.anaerobe.2015.11.002}{10.1016/j.anaerobe.2015.11.002}.

\leavevmode\hypertarget{ref-dieterle2019}{}%
6. \textbf{Dieterle MG}, \textbf{Rao K}, \textbf{Young VB}. 2018. Novel
therapies and preventative strategies for primary and recurrent
Clostridium difficile infections. Annals of the New York Academy of
Sciences \textbf{1435}:110--138.
doi:\href{https://doi.org/10.1111/nyas.13958}{10.1111/nyas.13958}.

\leavevmode\hypertarget{ref-juul2018}{}%
7. \textbf{Juul FE}, \textbf{Garborg K}, \textbf{Bretthauer M},
\textbf{Skudal H}, \textbf{Øines MN}, \textbf{Wiig H}, \textbf{Rose},
\textbf{Seip B}, \textbf{Lamont JT}, \textbf{Midtvedt T}, \textbf{Valeur
J}, \textbf{Kalager M}, \textbf{Holme}, \textbf{Helsingen L},
\textbf{Løberg M}, \textbf{Adami H-O}. 2018. Fecal microbiota
transplantation for primary clostridium difficile infection. New England
Journal of Medicine \textbf{378}:2535--2536.
doi:\href{https://doi.org/10.1056/nejmc1803103}{10.1056/nejmc1803103}.

\leavevmode\hypertarget{ref-seekatz2014}{}%
8. \textbf{Seekatz AM}, \textbf{Aas J}, \textbf{Gessert CE},
\textbf{Rubin TA}, \textbf{Saman DM}, \textbf{Bakken JS}, \textbf{Young
VB}. 2014. Recovery of the gut microbiome following fecal microbiota
transplantation. mBio \textbf{5}.
doi:\href{https://doi.org/10.1128/mbio.00893-14}{10.1128/mbio.00893-14}.

\leavevmode\hypertarget{ref-patron2017}{}%
9. \textbf{Patron RL}, \textbf{Hartmann CA}, \textbf{Allen S},
\textbf{Griesbach CL}, \textbf{Kosiorek HE}, \textbf{DiBaise JK},
\textbf{Orenstein R}. 2017. Vancomycin taper and risk of failure of
fecal microbiota transplantation in patients with recurrent clostridium
difficile infection. Clinical Infectious Diseases
\textbf{65}:1214--1217.
doi:\href{https://doi.org/10.1093/cid/cix511}{10.1093/cid/cix511}.

\leavevmode\hypertarget{ref-defilipp2019}{}%
10. \textbf{DeFilipp Z}, \textbf{Bloom PP}, \textbf{Soto MT},
\textbf{Mansour MK}, \textbf{Sater MRA}, \textbf{Huntley MH},
\textbf{Turbett S}, \textbf{Chung RT}, \textbf{Chen Y-B},
\textbf{Hohmann EL}. 2019. Drug-resistant e. Coli bacteremia transmitted
by fecal microbiota transplant. New England Journal of Medicine
\textbf{381}:2043--2050.
doi:\href{https://doi.org/10.1056/nejmoa1910437}{10.1056/nejmoa1910437}.

\leavevmode\hypertarget{ref-buffie2015}{}%
11. \textbf{Buffie CG}, \textbf{Bucci V}, \textbf{Stein RR},
\textbf{McKenney PT}, \textbf{Ling L}, \textbf{Gobourne A}, \textbf{No
D}, \textbf{Liu H}, \textbf{Kinnebrew M}, \textbf{Viale A},
\textbf{Littmann E}, \textbf{Brink MRM van den}, \textbf{Jenq RR},
\textbf{Taur Y}, \textbf{Sander C}, \textbf{Cross JR}, \textbf{Toussaint
NC}, \textbf{Xavier JB}, \textbf{Pamer EG}. 2014. Precision microbiome
reconstitution restores bile acid mediated resistance to clostridium
difficile. Nature \textbf{517}:205--208.
doi:\href{https://doi.org/10.1038/nature13828}{10.1038/nature13828}.

\leavevmode\hypertarget{ref-fletcher2018}{}%
12. \textbf{Fletcher JR}, \textbf{Erwin S}, \textbf{Lanzas C},
\textbf{Theriot CM}. 2018. Shifts in the gut metabolome and clostridium
difficile transcriptome throughout colonization and infection in a mouse
model. mSphere \textbf{3}.
doi:\href{https://doi.org/10.1128/msphere.00089-18}{10.1128/msphere.00089-18}.

\leavevmode\hypertarget{ref-reed2020}{}%
13. \textbf{Reed AD}, \textbf{Nethery MA}, \textbf{Stewart A},
\textbf{Barrangou R}, \textbf{Theriot CM}. 2020. Strain-dependent
inhibition of clostridioides difficile by commensal clostridia carrying
the bile acid-inducible (bai) operon. Journal of Bacteriology
\textbf{202}.
doi:\href{https://doi.org/10.1128/jb.00039-20}{10.1128/jb.00039-20}.

\leavevmode\hypertarget{ref-jenior2017}{}%
14. \textbf{Jenior ML}, \textbf{Leslie JL}, \textbf{Young VB},
\textbf{Schloss PD}. 2017. Clostridium difficile colonizes alternative
nutrient niches during infection across distinct murine gut microbiomes.
mSystems \textbf{2}.
doi:\href{https://doi.org/10.1128/msystems.00063-17}{10.1128/msystems.00063-17}.

\leavevmode\hypertarget{ref-lawley2012}{}%
15. \textbf{Lawley TD}, \textbf{Clare S}, \textbf{Walker AW},
\textbf{Stares MD}, \textbf{Connor TR}, \textbf{Raisen C},
\textbf{Goulding D}, \textbf{Rad R}, \textbf{Schreiber F},
\textbf{Brandt C}, \textbf{Deakin LJ}, \textbf{Pickard DJ},
\textbf{Duncan SH}, \textbf{Flint HJ}, \textbf{Clark TG},
\textbf{Parkhill J}, \textbf{Dougan G}. 2012. Targeted restoration of
the intestinal microbiota with a simple, defined bacteriotherapy
resolves relapsing clostridium difficile disease in mice. PLoS Pathogens
\textbf{8}:e1002995.
doi:\href{https://doi.org/10.1371/journal.ppat.1002995}{10.1371/journal.ppat.1002995}.

\leavevmode\hypertarget{ref-mcdonald2018}{}%
16. \textbf{McDonald JAK}, \textbf{Mullish BH}, \textbf{Pechlivanis A},
\textbf{Liu Z}, \textbf{Brignardello J}, \textbf{Kao D}, \textbf{Holmes
E}, \textbf{Li JV}, \textbf{Clarke TB}, \textbf{Thursz MR},
\textbf{Marchesi JR}. 2018. Inhibiting growth of clostridioides
difficile by restoring valerate, produced by the intestinal microbiota.
Gastroenterology \textbf{155}:1495--1507.e15.
doi:\href{https://doi.org/10.1053/j.gastro.2018.07.014}{10.1053/j.gastro.2018.07.014}.

\leavevmode\hypertarget{ref-ghimire2019}{}%
17. \textbf{Ghimire S}, \textbf{Roy C}, \textbf{Wongkuna S},
\textbf{Antony L}, \textbf{Maji A}, \textbf{Keena MC}, \textbf{Foley A},
\textbf{Scaria J}. 2020. Identification of clostridioides
difficile-inhibiting gut commensals using culturomics, phenotyping, and
combinatorial community assembly. mSystems \textbf{5}.
doi:\href{https://doi.org/10.1128/msystems.00620-19}{10.1128/msystems.00620-19}.

\leavevmode\hypertarget{ref-auchtung2020}{}%
18. \textbf{Auchtung JM}, \textbf{Preisner EC}, \textbf{Collins J},
\textbf{Lerma AI}, \textbf{Britton RA}. 2020. Identification of
simplified microbial communities that inhibit clostridioides difficile
infection through dilution/extinction. mSphere \textbf{5}.
doi:\href{https://doi.org/10.1128/msphere.00387-20}{10.1128/msphere.00387-20}.

\leavevmode\hypertarget{ref-schubert2015}{}%
19. \textbf{Schubert AM}, \textbf{Sinani H}, \textbf{Schloss PD}. 2015.
Antibiotic-induced alterations of the murine gut microbiota and
subsequent effects on colonization resistance against clostridium
difficile. mBio \textbf{6}.
doi:\href{https://doi.org/10.1128/mbio.00974-15}{10.1128/mbio.00974-15}.

\leavevmode\hypertarget{ref-jenior2018}{}%
20. \textbf{Jenior ML}, \textbf{Leslie JL}, \textbf{Young VB},
\textbf{Schloss PD}. 2018. Clostridium difficile alters the structure
and metabolism of distinct cecal microbiomes during initial infection to
promote sustained colonization. mSphere \textbf{3}.
doi:\href{https://doi.org/10.1128/msphere.00261-18}{10.1128/msphere.00261-18}.

\leavevmode\hypertarget{ref-kurtz2015}{}%
21. \textbf{Kurtz ZD}, \textbf{Müller CL}, \textbf{Miraldi ER},
\textbf{Littman DR}, \textbf{Blaser MJ}, \textbf{Bonneau RA}. 2015.
Sparse and compositionally robust inference of microbial ecological
networks. PLOS Computational Biology \textbf{11}:e1004226.
doi:\href{https://doi.org/10.1371/journal.pcbi.1004226}{10.1371/journal.pcbi.1004226}.

\leavevmode\hypertarget{ref-reeves2011}{}%
22. \textbf{Reeves AE}, \textbf{Theriot CM}, \textbf{Bergin IL},
\textbf{Huffnagle GB}, \textbf{Schloss PD}, \textbf{Young VB}. 2011. The
interplay between microbiome dynamics and pathogen dynamics in a murine
model ofClostridium difficileInfection. Gut Microbes
\textbf{2}:145--158.
doi:\href{https://doi.org/10.4161/gmic.2.3.16333}{10.4161/gmic.2.3.16333}.

\leavevmode\hypertarget{ref-reeves2012}{}%
23. \textbf{Reeves AE}, \textbf{Koenigsknecht MJ}, \textbf{Bergin IL},
\textbf{Young VB}. 2012. Suppression of clostridium difficile in the
gastrointestinal tracts of germfree mice inoculated with a murine
isolate from the family lachnospiraceae. Infection and Immunity
\textbf{80}:3786--3794.
doi:\href{https://doi.org/10.1128/iai.00647-12}{10.1128/iai.00647-12}.

\leavevmode\hypertarget{ref-tomkovich2020}{}%
24. \textbf{Tomkovich S}, \textbf{Stough JMA}, \textbf{Bishop L},
\textbf{Schloss PD}. 2020. The initial gut microbiota and response to
antibiotic perturbation influence clostridioides difficile clearance in
mice. mSphere \textbf{5}.
doi:\href{https://doi.org/10.1128/msphere.00869-20}{10.1128/msphere.00869-20}.

\leavevmode\hypertarget{ref-peterfreund2012}{}%
25. \textbf{Peterfreund GL}, \textbf{Vandivier LE}, \textbf{Sinha R},
\textbf{Marozsan AJ}, \textbf{Olson WC}, \textbf{Zhu J}, \textbf{Bushman
FD}. 2012. Succession in the gut microbiome following antibiotic and
antibody therapies for clostridium difficile. PLoS ONE
\textbf{7}:e46966.
doi:\href{https://doi.org/10.1371/journal.pone.0046966}{10.1371/journal.pone.0046966}.

\leavevmode\hypertarget{ref-fukuyama2017}{}%
26. \textbf{Fukuyama J}, \textbf{Rumker L}, \textbf{Sankaran K},
\textbf{Jeganathan P}, \textbf{Dethlefsen L}, \textbf{Relman DA},
\textbf{Holmes SP}. 2017. Multidomain analyses of a longitudinal human
microbiome intestinal cleanout perturbation experiment. PLOS
Computational Biology \textbf{13}:e1005706.
doi:\href{https://doi.org/10.1371/journal.pcbi.1005706}{10.1371/journal.pcbi.1005706}.

\leavevmode\hypertarget{ref-suez2018}{}%
27. \textbf{Suez J}, \textbf{Zmora N}, \textbf{Zilberman-Schapira G},
\textbf{Mor U}, \textbf{Dori-Bachash M}, \textbf{Bashiardes S},
\textbf{Zur M}, \textbf{Regev-Lehavi D}, \textbf{Brik RB-Z},
\textbf{Federici S}, \textbf{Horn M}, \textbf{Cohen Y}, \textbf{Moor
AE}, \textbf{Zeevi D}, \textbf{Korem T}, \textbf{Kotler E},
\textbf{Harmelin A}, \textbf{Itzkovitz S}, \textbf{Maharshak N},
\textbf{Shibolet O}, \textbf{Pevsner-Fischer M}, \textbf{Shapiro H},
\textbf{Sharon I}, \textbf{Halpern Z}, \textbf{Segal E}, \textbf{Elinav
E}. 2018. Post-antibiotic gut mucosal microbiome reconstitution is
impaired by probiotics and improved by autologous FMT. Cell
\textbf{174}:1406--1423.e16.
doi:\href{https://doi.org/10.1016/j.cell.2018.08.047}{10.1016/j.cell.2018.08.047}.

\leavevmode\hypertarget{ref-ianiro2018}{}%
28. \textbf{Ianiro G}, \textbf{Maida M}, \textbf{Burisch J},
\textbf{Simonelli C}, \textbf{Hold G}, \textbf{Ventimiglia M},
\textbf{Gasbarrini A}, \textbf{Cammarota G}. 2018. Efficacy of different
faecal microbiota transplantation protocols for clostridium difficile
infection: A systematic review and meta-analysis. United European
Gastroenterology Journal \textbf{6}:1232--1244.
doi:\href{https://doi.org/10.1177/2050640618780762}{10.1177/2050640618780762}.

\leavevmode\hypertarget{ref-allegretti2020}{}%
29. \textbf{Allegretti JR}, \textbf{Mehta SR}, \textbf{Kassam Z},
\textbf{Kelly CR}, \textbf{Kao D}, \textbf{Xu H}, \textbf{Fischer M}.
2020. Risk factors that predict the failure of multiple fecal microbiota
transplantations for clostridioides difficile infection. Digestive
Diseases and Sciences.
doi:\href{https://doi.org/10.1007/s10620-020-06198-2}{10.1007/s10620-020-06198-2}.

\leavevmode\hypertarget{ref-garzagonzlez2019}{}%
30. \textbf{Garza-González E}, \textbf{Mendoza-Olazarán S},
\textbf{Morfin-Otero R}, \textbf{Ramírez-Fontes A},
\textbf{Rodríguez-Zulueta P}, \textbf{Flores-Treviño S},
\textbf{Bocanegra-Ibarias P}, \textbf{Maldonado-Garza H},
\textbf{Camacho-Ortiz A}. 2019. Intestinal microbiome changes in fecal
microbiota transplant (FMT) vs. FMT enriched with lactobacillus in the
treatment of recurrent clostridioides difficile infection. Canadian
Journal of Gastroenterology and Hepatology \textbf{2019}:1--7.
doi:\href{https://doi.org/10.1155/2019/4549298}{10.1155/2019/4549298}.

\leavevmode\hypertarget{ref-winter2013}{}%
31. \textbf{Winter SE}, \textbf{Lopez CA}, \textbf{Bäumler AJ}. 2013.
The dynamics of gut-associated microbial communities during
inflammation. EMBO reports \textbf{14}:319--327.
doi:\href{https://doi.org/10.1038/embor.2013.27}{10.1038/embor.2013.27}.

\leavevmode\hypertarget{ref-riverachavez2017}{}%
32. \textbf{Rivera-Chávez F}, \textbf{Lopez CA}, \textbf{Bäumler AJ}.
2017. Oxygen as a driver of gut dysbiosis. Free Radical Biology and
Medicine \textbf{105}:93--101.
doi:\href{https://doi.org/10.1016/j.freeradbiomed.2016.09.022}{10.1016/j.freeradbiomed.2016.09.022}.

\leavevmode\hypertarget{ref-carlson2013}{}%
33. \textbf{Carlson PE}, \textbf{Walk ST}, \textbf{Bourgis AET},
\textbf{Liu MW}, \textbf{Kopliku F}, \textbf{Lo E}, \textbf{Young VB},
\textbf{Aronoff DM}, \textbf{Hanna PC}. 2013. The relationship between
phenotype, ribotype, and clinical disease in human clostridium difficile
isolates. Anaerobe \textbf{24}:109--116.
doi:\href{https://doi.org/10.1016/j.anaerobe.2013.04.003}{10.1016/j.anaerobe.2013.04.003}.

\leavevmode\hypertarget{ref-thanissery2017}{}%
34. \textbf{Thanissery R}, \textbf{Winston JA}, \textbf{Theriot CM}.
2017. Inhibition of spore germination, growth, and toxin activity of
clinically relevant c.~difficile strains by gut microbiota derived
secondary bile acids. Anaerobe \textbf{45}:86--100.
doi:\href{https://doi.org/10.1016/j.anaerobe.2017.03.004}{10.1016/j.anaerobe.2017.03.004}.

\leavevmode\hypertarget{ref-theriot2011}{}%
35. \textbf{Theriot CM}, \textbf{Koumpouras CC}, \textbf{Carlson PE},
\textbf{Bergin II}, \textbf{Aronoff DM}, \textbf{Young VB}. 2011.
Cefoperazone-treated mice as an experimental platform to assess
differential virulence ofClostridium difficilestrains. Gut Microbes
\textbf{2}:326--334.
doi:\href{https://doi.org/10.4161/gmic.19142}{10.4161/gmic.19142}.

\leavevmode\hypertarget{ref-rao2015}{}%
36. \textbf{Rao K}, \textbf{Micic D}, \textbf{Natarajan M},
\textbf{Winters S}, \textbf{Kiel MJ}, \textbf{Walk ST}, \textbf{Santhosh
K}, \textbf{Mogle JA}, \textbf{Galecki AT}, \textbf{LeBar W},
\textbf{Higgins PDR}, \textbf{Young VB}, \textbf{Aronoff DM}. 2015.
Clostridium difficile Ribotype 027: Relationship to age, detectability
of toxins a or b in stool with rapid testing, severe infection, and
mortality. Clinical Infectious Diseases \textbf{61}:233--241.
doi:\href{https://doi.org/10.1093/cid/civ254}{10.1093/cid/civ254}.

\leavevmode\hypertarget{ref-sorg2009}{}%
37. \textbf{Sorg JA}, \textbf{Dineen SS}. 2009. Laboratory maintenance
ofClostridium difficile. Current Protocols in Microbiology \textbf{12}.
doi:\href{https://doi.org/10.1002/9780471729259.mc09a01s12}{10.1002/9780471729259.mc09a01s12}.

\leavevmode\hypertarget{ref-winston2016}{}%
38. \textbf{Winston JA}, \textbf{Thanissery R}, \textbf{Montgomery SA},
\textbf{Theriot CM}. 2016. Cefoperazone-treated mouse model of
clinically-relevant \(\Delta\)clostridium difficile strain r20291.
Journal of Visualized Experiments.
doi:\href{https://doi.org/10.3791/54850}{10.3791/54850}.

\leavevmode\hypertarget{ref-kozich2013}{}%
39. \textbf{Kozich JJ}, \textbf{Westcott SL}, \textbf{Baxter NT},
\textbf{Highlander SK}, \textbf{Schloss PD}. 2013. Development of a
dual-index sequencing strategy and curation pipeline for analyzing
amplicon sequence data on the MiSeq illumina sequencing platform.
Applied and Environmental Microbiology \textbf{79}:5112--5120.
doi:\href{https://doi.org/10.1128/aem.01043-13}{10.1128/aem.01043-13}.

\leavevmode\hypertarget{ref-schloss2009}{}%
40. \textbf{Schloss PD}, \textbf{Westcott SL}, \textbf{Ryabin T},
\textbf{Hall JR}, \textbf{Hartmann M}, \textbf{Hollister EB},
\textbf{Lesniewski RA}, \textbf{Oakley BB}, \textbf{Parks DH},
\textbf{Robinson CJ}, \textbf{Sahl JW}, \textbf{Stres B},
\textbf{Thallinger GG}, \textbf{Horn DJV}, \textbf{Weber CF}. 2009.
Introducing mothur: Open-source, platform-independent,
community-supported software for describing and comparing microbial
communities. Applied and Environmental Microbiology
\textbf{75}:7537--7541.
doi:\href{https://doi.org/10.1128/aem.01541-09}{10.1128/aem.01541-09}.

\leavevmode\hypertarget{ref-wang2007}{}%
41. \textbf{Wang Q}, \textbf{Garrity GM}, \textbf{Tiedje JM},
\textbf{Cole JR}. 2007. Naïve bayesian classifier for rapid assignment
of rRNA sequences into the new bacterial taxonomy. Applied and
Environmental Microbiology \textbf{73}:5261--5267.
doi:\href{https://doi.org/10.1128/aem.00062-07}{10.1128/aem.00062-07}.

\leavevmode\hypertarget{ref-yue2005}{}%
42. \textbf{Yue JC}, \textbf{Clayton MK}. 2005. A similarity measure
based on species proportions. Communications in Statistics - Theory and
Methods \textbf{34}:2123--2131.
doi:\href{https://doi.org/10.1080/sta-200066418}{10.1080/sta-200066418}.

\leavevmode\hypertarget{ref-benjamini1995}{}%
43. \textbf{Benjamini Y}, \textbf{Hochberg Y}. 1995. Controlling the
false discovery rate: A practical and powerful approach to multiple
testing. Journal of the Royal Statistical Society: Series B
(Methodological) \textbf{57}:289--300.
doi:\href{https://doi.org/10.1111/j.2517-6161.1995.tb02031.x}{10.1111/j.2517-6161.1995.tb02031.x}.

\leavevmode\hypertarget{ref-topcuoglu2020}{}%
44. \textbf{Topçuoğlu BD}, \textbf{Lesniak NA}, \textbf{Ruffin MT},
\textbf{Wiens J}, \textbf{Schloss PD}. 2020. A framework for effective
application of machine learning to microbiome-based classification
problems. mBio \textbf{11}.
doi:\href{https://doi.org/10.1128/mbio.00434-20}{10.1128/mbio.00434-20}.
\end{cslreferences}

\newpage

\includegraphics{../results/figures/figure_1.jpg}

\textbf{Figure 1. Reduced antibiotic doses permitted murine communities
to be colonized and spontaneously clear that \emph{C. difficile}
colonization.} (A-C) Daily CFU of \emph{C. difficile} in fecal samples
of mice treated with clindamycin, cefoperazone, or streptomycin from
time of challenge (Day 0) through 10 days post infection (dpi). The bold
line is the median CFU of the group and the transparent lines are the
individual mice. (D-F) Relative abundance of twelve most abundant genera
at the time of \emph{C. difficile} challenge, all other genera grouped
into Other. Each column is an individual mouse. LOD = Limit of
detection. (clindamycin - 10 mg/kg N =11; cefoperazone - 0.5 mg/mL N =
5, 0.3 mg/mL N = 9, 0.1 mg/mL N = 2; streptomycin - 5.0 mg/mL N = 8, 0.5
mg/mL N = 7, 0.1 mg/mL N = 7).

\hfill\break

\includegraphics{../results/figures/figure_2.jpg}

\textbf{Figure 2. Microbiota community diversity showed
antibiotic-specific trends associated with \emph{C. difficle}
colonization clearance.} For communities colonized with \emph{C.
difficile} from mice treated with clindamycin (A), cefoperazone (B), and
streptomycin (C), microbiota \(\alpha\)-diversity (S\textsubscript{obs}
and Inverse Simpson) and \(\beta\)-diversity
(\(\theta\)\textsubscript{YC}) were compared at the initial
pre-antibiotic treatment state, time of \emph{C. difficile} challenge
(TOC), and end of the experiment. \(\beta\)-diversity
(\(\theta\)\textsubscript{YC}) was compared between the initial
pre-antibiotic treatment to all other initial pre-antibiotic treatment
communities treated with the same antibiotic, the initial community to
the same community at the time of \emph{C. difficile} challenge, and the
initial community to the same community at end of the experiment.
(clindamycin - cleared N = 11; cefoperazone - cleared N = 7, colonized N
= 9; streptomycin - cleared N = 9, colonized N = 11). * indicates
statistical significance of \emph{P} \textless{} 0.05, calculated by
Wilcoxon rank sum test with Benjamini-Hochberg correction.

\hfill\break

\includegraphics{../results/figures/figure_3.jpg}

\textbf{Figure 3. OTU abundance differences between communities that
cleared \emph{C. difficile} colonization and remained colonized are
unique to each treatment.} For cefoperazone (A) and streptomycin (B),
the difference in the relative abundance of OTUs that were significantly
different between communities that eliminated \emph{C. difficile}
colonization and those that remained colonized within each antibiotic
treatment for each time point. Bold points are median relative abundance
and transparent points are relative abundance of individual mice. Lines
connect points within each comparison to show difference in medians.
Only OTUs at time points with statistically significant differences,
\emph{P} \textless{} 0.05, were plotted (calculated by Wilcoxon rank sum
test with Benjamini-Hochberg correction). Limit of detection (LOD).

\hfill\break

\includegraphics{../results/figures/figure_4.jpg}

\textbf{Figure 4. Each antibiotic had specific sets of temporal changes
in OTU abundance associated with \emph{C. difficile} colonization and
clearance.} For clindamycin (A), cefoperazone (C), and streptomycin (B,
D), the difference in the relative abundance of OTUs that were
significantly different between time points within each \emph{C.
difficile} colonization outcome for each antibiotic treatment. Bold
points are median relative abundance and transparent points are relative
abundance of individual mice. Lines connect points within each
comparison to show difference in medians. Arrows point in the direction
of the temporal change of the relative abundance. Only OTUs at time
points with statistically significant differences, \emph{P} \textless{}
0.05, were plotted (calculated by Wilcoxon rank sum test with
Benjamini-Hochberg correction). Bold OTUs were shared across outcomes.
Limit of detection (LOD).

\hfill\break

\includegraphics{../results/figures/figure_5.jpg}

\textbf{Figure 5. Distinct features of the bacterial community at the
time of infection can classify end point colonization.} (A) L2 logistic
regression model features' importance determined by the decrease in
model performance when randomizing an individual feature. All OTUs
affecting performance shown. Dashed lines show performance range of
final model with all features included. (B) Distribution of odds ratio
used in L2 logistic regression model. Values above 1 indicate abundance
predicted the community cleared colonization (red) and values below 1
indicate abundance predicted \emph{C. difficile} remained colonized
(blue). Feature label and boxplot are colored to match the median odds
ratio. (C) Relative abundance difference in features used by L2 logistic
regression model displayed by antibiotic treatment.

\hfill\break

\includegraphics{../results/figures/figure_6.jpg}

\textbf{Figure 6. Conditional independence networks reveal
treatment-specific relationships between the community and \emph{C.
difficile} during colonization clearance.} (A) SPIEC-EASI (sparse
inverse covariance estimation for ecological association inference)
networks showing conditionally independent first-order relationships
between \emph{C. difficile} and the community as \emph{C. difficile} was
cleared from the gut environment. Nodes are sized by median relative
abundance of the OTU. A red colored edge indicates a negative
interaction and blue indicates a positive interaction, while edge
thickness indicates the interaction strength. (B) Network centrality
measured with betweenness, i.e.~how many paths between two OTUs pass
through an individual, and degree, i.e.~how many connections an OTU had.
* indicates statistical significance of \emph{P} \textless{} 0.05,
calculated by Wilcoxon rank sum test with Benjamini-Hochberg correction.

\hfill\break

\includegraphics{../results/figures/figure_S1.jpg}

\textbf{Figure S1. Initial microbiota relative abundance of mice prior
to antibiotic treatment.} Relative abundance at the beginning of the
experiment prior to antibiotic treatment of twelve most abundant genera
post antibiotic treatment, all other genera grouped into Other. Each
column is an individual mouse. Color intensity is
log\textsubscript{10}-transformed mean percent relative abundance of
each day. (N = 57).

\hfill\break

\includegraphics{../results/figures/figure_S2.jpg}

\textbf{Figure S2. Temporally differing OTU for cefoperazone-treated
mice that cleared \emph{C. difficile} colonization.} Bold points are
median relative abundance and transparent points are relative abundance
of individual mice. Lines connect points within each comparison to show
difference in medians. Arrows point in the direction of the temporal
change of the relative abundance. Only OTUs at time points with
statistically significant differences, \emph{P} \textless{} 0.05, were
plotted (calculated by Wilcoxon rank sum test with Benjamini-Hochberg
correction). Limit of detection (LOD).

\hfill\break

\includegraphics{../results/figures/figure_S3.jpg}

\textbf{Figure S3. Bacterial community at the time of infection can
classify endpoint colonization.} Classification performance of L2
logistic regression. Area under the receiver-operator curve for
classifying if the community will remain colonized based on the OTUs
present at the time of \emph{C. difficile} infection (Day 0).
Cross-validation of model performed on half of the data to tune model
(CV AUC) and then tuned model was tested on the held-out data (Test
AUC).


\end{document}
